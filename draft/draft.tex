\documentclass[11pt]{article}
\usepackage{graphicx}
\usepackage{multirow}
\usepackage{mathtools}
\usepackage{amsmath,amsthm,amssymb, amsfonts}
\usepackage{relsize}
\usepackage{graphicx}
\graphicspath{ {./} }
\usepackage[a4paper, total={6in, 8in}]{geometry}

%description -> general document format:
\newcommand{\boldtext}[1]{\textbf{#1}}

\newcommand{\image}[2]{\includegraphics[scale=#1]{#2}}

\newcommand{\mysection}[1]{
    \section*{#1}
    \hrule
    \vspace*{0.5cm}
}

\newcommand{\mysubsection}[1]{
    \subsection*{· #1}
}

%description -> general math format:
\newcommand{\myunderscore}[2]{
    \underbrace{#1}_\text{#2}
}

\newcommand{\twolinesplit}[3]{
    \begin{equation*}
        #1
        \begin{cases}
             & #2   \\
             & #3
        \end{cases}
    \end{equation*}
}

\newcommand{\threelinesplit}[4]{
    \begin{equation*}
        #1
        \begin{cases}
             & #2   \\
             & #3   \\
             & #4
        \end{cases}
    \end{equation*}
}

%description -> discrete mathematics:
\newcommand{\R}{\mathbb{R}} % Real numbers
\newcommand{\N}{\mathbb{N}} % Natural numbers
\newcommand{\Z}{\mathbb{Z}} % Integers
\newcommand{\Q}{\mathbb{Q}} % Rational numbers
\newcommand{\set}[1]{\{#1\}} % items wrapped in a set
\newcommand{\powerset}[1]{\mathcal{P}(#1)} % items wrapped in a powerset
\newcommand{\floor}[1]{\lfloor #1 \rfloor} % floor notation
\newcommand{\floorfrac}[2]{\lfloor \cfrac{#1}{#2} \rfloor} % floor notation but with a fraction inside
\newcommand{\ceil}[1]{\lceil #1 \rceil} % ceiling notation
\newcommand{\ceilfrac}[2]{\lceil \cfrac{#1}{#2} \rceil} % ceiling notation but with a fraction inside
\newcommand{\Lsum}[2]{\mathlarger{\sum}_{#1}^{#2}} % a larger sigma notation
\newcommand{\hfrac}[2]{\left(\cfrac{#1}{#2}\right)} % a "high" fraction where parathasis's height is adjusted to fit the fraction
\newcommand{\RR}[1]{\mathcal{R}_#1} % testing -> Relational (temporal)
\renewcommand{\R}{\mathcal{R}} % testing -> Relational (temporal)

%description -> linear algebra:

\renewcommand{\det}[1]{
    $\begin{vmatrix*}[r] #1 \end{vmatrix*}$
}

\newcommand{\mateq}{$\backsim$}
\newcommand{\mat}[1]{
    $\begin{bmatrix*}[r]
        #1
    \end{bmatrix*}$
} % general matrix

\newcommand{\tomat}[2]{
    $\begin{bmatrix*}[r]
        #1 \\
        #2
    \end{bmatrix*}$
} % two-one matrix

\newcommand{\ttmat}[4]{
    $\begin{bmatrix*}[r]
        #1 & #2 \\
        #3 & #4
    \end{bmatrix*}$
} % two-two matrix

\newcommand{\thmat}[3]{
    $\begin{bmatrix*}[r] 
        #1 \\ 
        #2 \\
        #3 
    \end{bmatrix*}$
} % three-three matrix (line combined form) (specific element in line omitted)

\newcommand{\mytable}[2]{
    \begin{center}
        \begin{tabular}{#1}
            #2
        \end{tabular}
    \end{center}
}

\newcommand{\myalign}[1]{\begin{align*}#1\end{align*}}
\newcommand{\gap}{\vspace*{0.5cm}}
\newcommand{\linearsec}[1]{
    \section*{Question #1}
    \hrule
    \vspace*{0.5cm}
}

\title{Geometry Lab - Computer Graphics \& Robotics - Sept 20 - Blog}
\author{David Cao}
\date{Sept 20}

\begin{document}

\maketitle

\section*{Forward Kinematic Problem \#1}
\section{Problem statement:}
Given a robot with a list of of segments, joints, and prismatic joint, can we give an explicit description of the areas that the robot can reach:
\vspace{3em} \\
\image{0.75}{img1}\\
\begin{center}
  Example 1
\end{center}

\section{Reading summary:}
§2 of the reading introduces a new representation of the reachable position using linear algebra. \\
\\
First, at At a revolute joint $i$, we introduce an $(x_{i+1}, y_{i+1})$ coordinate system in the following way:
We take the positive xi+1-axis to lie along
the direction of segment i + 1 (in the robot's current position). Then the positive
$y_{i+1}$-axis is determined to form a normal right-handed rectangular coordinate system. A illustration of this local coordinate system is shown below:
\image{0.6}{img2} \\
Giving the angle between them $\theta_i$ and length of the segment $l_i$,
such representation allows us to relate coordinate $(x_{i+1}, y_{i+1})$ with $(x_i, y_i)$ using linear algebra:
\begin{center}
  \mat{
    x_i \\
    y_1
  } = \mat{
    cos(\theta_i) & -sin(\theta_i) \\
    sin(\theta_i) & cos(\theta_i) \\
  } $\cdot$ \mat{
    x_{i+1} \\
    y_{i+1}
  } + \mat{
    l_i \\
    0
  }
\end{center}
Or better using 3 by 3 matrix:
\begin{center}
  \mat{
    x_i \\
    y_1 \\
    1
  } = \mat{
    cos(\theta_i) & -sin(\theta_i) & l_i \\
    sin(\theta_i) & cos(\theta_i) & 0 \\
    0 & 0 & 1
  } $\cdot$ \mat{
    x_{i+1} \\
    y_{i+1} \\
    1
  } = $A_i \cdot$ \mat{
    x_{i+1} \\
    y_{i+1} \\
    1
  }
\end{center}
Matrix notation allows us to express the final coordinate in a more concise way:
\begin{center}
  \mat{
    x_i \\
    y_1 \\
    1
  } = $A_1A_2A_3 \cdot$ \mat{
    x_{i+1} \\
    y_{i+1} \\
    1
  } \\
\end{center}
Using the trigonometric addition formulas, this equation can be written as
\begin{center}
  \mat{
    x_i \\
    y_1 \\
    1
  } =
  \mat{
    cos(\theta_1+\theta_2+\theta_3) & -sin(\theta_1+\theta_2+\theta_3) & l_3cos(\theta_1+\theta_2) + l_2cos(\theta_1) \\
    sin(\theta_1+\theta_2+\theta_3) & cos(\theta_1+\theta_2+\theta_3) & l_3sin(\theta_1+\theta_2) + l_2sin(\theta_1) \\
    0 & 0 & 1
  }
  $\cdot$ \mat{
    x_{4} \\
    y_{4} \\
    1
  }
\end{center}
When calculating the matrix product, we can assume that the end point of the robot $(x_4, y_4)$ is the origin,
and through matrix multiplication get the relative coordinate of the point $(x_4, y_4)$ with respect to $(x_0, y_0)$.
Since we can assume that $x_0 = y_0 = 0$, the result of the matrix multiplication is:
\begin{center}
  \mat{
    x_1 \\
    y_1 \\
    1
  } = \mat{
    & l_3cos(\theta_1 + \theta_2) + l_2cos(\theta_1) \\
    & l_3sin(\theta_1 + \theta_2) + l_2sin(\theta_1) \\
    & 1
  } \\
\end{center}
Besides using $\theta$ and $l$, using parametrization $x = cos(\theta)$ and $y = sin(\theta)$, we can represent the result in a polynomial form:
\begin{center}
  \mat{
    l_3(c_1c_2-s_1s_2) + l_2c_1\\
    l_3(s_1c_2+s_2c_1) + l_2s_1
  }
\end{center}
Moreover, we can also use rational parametrization to represent reach:
\myalign{
  & x = \cfrac{1-t^2}{1+t^2} \\
  & y = \cfrac{2}{1+t^2}
}
\section{Exercise Problem:}
\subsection{Problem 1 statement:}
We claim that \\
\begin{center}
  \mat{
    x_1 \\
    y_1
  } = \mat{
    cos(\theta) & -sin(\theta) \\
    sin(\theta) & cos(\theta)
  } $\cdot$ \mat{
    x_2 \\
    y_2
  }
\end{center}
Define:
\begin{center}
  $q = (x_2, y_2) = (rcos(\alpha), rsin(\alpha))$
\end{center}
Show that:
\begin{center}
  $q = (x_1, x_2) = (rcos(\alpha + \theta), rsin(\alpha + \theta))$
\end{center}
And prove the matrix multiplication
\subsection*{Problem 2 statement:}
Show that any affine transformation in the plane
\myalign{
  x^{'} = ax + by + e \\
  y^{'} = cx + dy + f
}
can be represented in a similar way:
\begin{center}
  \mat{
    x^{'} \\
    y^{'} \\
    1
  } = \mat{
    a & b & e \\
    c & d & f \\
    0 & 0 & 1
  } $\cdot$ \mat{
    x \\
    y \\
    1
  }
\end{center}
Give a similar representation for affine transformations of $\mathbb{R}^3$ using 4 × 4 matrices.

\section{Discussion and discovery:}
\subsection{Problem 1:}
Looking at the illustrative graph below, we can notice that under $(x_2, y_2)$ coordinate, we can write
the point as:
\begin{center}
  $(rcos(\alpha), rsin(\alpha))$
\end{center}
But if we rotate the axis by an angle of $\alpha$, we would also need to add the angle $\alpha$ in the polar coordinate, resulting in $(x_1, y_1)$ to be represented as:
\begin{center}
  $(rcos(\theta + \alpha), rsin(\theta + \alpha))$
\end{center}
\image{0.22}{img4} \\
Now to prove the rotational matrix, we can set the equation to:
\begin{center}
  \mat{
    rcos(\theta + \alpha) \\
    rsin(\theta + \alpha)
  } = \mat{
    x & y \\
    z & a
  } $\cdot$ \mat{
    rcos(\alpha) \\
    rsin(\alpha)
  } \\
  \gap
  \mat{
    rcos(\theta + \alpha) \\
    rsin(\theta + \alpha)
  } = \mat{
    rxcos(\alpha) + rysin(\alpha) \\
    rzcos(\alpha) + r\alpha sin(\alpha)
  } \\
  \gap
  $cos(\theta + \alpha) = cos(\theta)cos(\alpha)-sin(\theta)sin(\alpha)$ \\
  $sin(\theta + \alpha) = sin(\theta)cos(\alpha)-sin(\theta)cos(\alpha)$ \\
  $x = cos(\theta)$ \\
  $y = -sin(\theta)$ \\
  $z = sin(\theta)$ \\
  $a = cos(\theta)$ \\
\end{center}
\subsection{Problem 2:}
The matrix form can be easily proved by completing the matrix multiplication, the similar representation for $\mathbb{R}^3$ can be represented as:
\begin{center}
  \mat{
    x^{'} \\
    y^{'} \\
    z^{'} \\
    1
  } = \mat{
    a & b & c & j \\
    d & e & f & k \\
    g & h & i & l \\
    0 & 0 & 0 & 1
  } $\cdot$ \mat{
    x \\
    y \\
    z \\
    1
  }
\end{center}
It is also important to talk about the reason behind using 3 by 3 matrix for 2D and 4 by 4 matrix for 3D. The reason is that the 3 by 3 matrix is used to represent the rotation and translation in 2D, while the 4 by 4 matrix is used to represent the rotation, translation, and scaling in 3D. \\
For example, if we want to represent both transformation and rotation in 2D, we will need this equation:
\begin{center}
  \mat{
    x_i \\
    b_i
  } = \mat{
    cos(\theta_i) & -sin(\theta_i)  \\
    sin(\theta_i) & cos(\theta_i) \\
  } $\cdot$ \mat{
    x_{i+1} \\
    y_{i+1} \\
  } + \mat{
    l_i \\
    0
  }
\end{center}
We notice that the vector addition for translation at the end could be hard to calculate when there are multiple transformations, however, this translation can be easily represented in the 3 by 3 matrix form:
\begin{center}
  \mat{
    x_i \\
    b_i \\
    1
  } = \mat{
    cos(\theta_i) & -sin(\theta_i) & l_i \\
    sin(\theta_i) & cos(\theta_i) & 0 \\
    0 & 0 & 1
  } $\cdot$ \mat{
    x_{i+1} \\
    y_{i+1} \\
    1
  }
\end{center}

\end{document}