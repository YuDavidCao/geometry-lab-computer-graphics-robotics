\documentclass[11pt]{article}
\usepackage{graphicx}
\usepackage{multirow}
\usepackage{mathtools}
\usepackage{amsmath,amsthm,amssymb, amsfonts}
\usepackage{relsize}
\usepackage{graphicx}
\graphicspath{ {./} }
\usepackage[a4paper, total={6in, 8in}]{geometry}

%description -> general document format:
\newcommand{\boldtext}[1]{\textbf{#1}}

\newcommand{\image}[2]{\includegraphics[scale=#1]{#2}}

\newcommand{\mysection}[1]{
    \section*{#1}
    \hrule
    \vspace*{0.5cm}
}

\newcommand{\mysubsection}[1]{
    \subsection*{· #1}
}

%description -> general math format:
\newcommand{\myunderscore}[2]{
    \underbrace{#1}_\text{#2}
}

\newcommand{\twolinesplit}[3]{
    \begin{equation*}
        #1
        \begin{cases}
             & #2   \\
             & #3
        \end{cases}
    \end{equation*}
}

\newcommand{\threelinesplit}[4]{
    \begin{equation*}
        #1
        \begin{cases}
             & #2   \\
             & #3   \\
             & #4
        \end{cases}
    \end{equation*}
}

%description -> discrete mathematics:
\newcommand{\R}{\mathbb{R}} % Real numbers
\newcommand{\N}{\mathbb{N}} % Natural numbers
\newcommand{\Z}{\mathbb{Z}} % Integers
\newcommand{\Q}{\mathbb{Q}} % Rational numbers
\newcommand{\set}[1]{\{#1\}} % items wrapped in a set
\newcommand{\powerset}[1]{\mathcal{P}(#1)} % items wrapped in a powerset
\newcommand{\floor}[1]{\lfloor #1 \rfloor} % floor notation
\newcommand{\floorfrac}[2]{\lfloor \cfrac{#1}{#2} \rfloor} % floor notation but with a fraction inside
\newcommand{\ceil}[1]{\lceil #1 \rceil} % ceiling notation
\newcommand{\ceilfrac}[2]{\lceil \cfrac{#1}{#2} \rceil} % ceiling notation but with a fraction inside
\newcommand{\Lsum}[2]{\mathlarger{\sum}_{#1}^{#2}} % a larger sigma notation
\newcommand{\hfrac}[2]{\left(\cfrac{#1}{#2}\right)} % a "high" fraction where parathasis's height is adjusted to fit the fraction
\newcommand{\RR}[1]{\mathcal{R}_#1} % testing -> Relational (temporal)
\renewcommand{\R}{\mathcal{R}} % testing -> Relational (temporal)

%description -> linear algebra:

\renewcommand{\det}[1]{
    $\begin{vmatrix*}[r] #1 \end{vmatrix*}$
}

\newcommand{\mateq}{$\backsim$}
\newcommand{\mat}[1]{
    $\begin{bmatrix*}[r]
        #1
    \end{bmatrix*}$
} % general matrix

\newcommand{\tomat}[2]{
    $\begin{bmatrix*}[r]
        #1 \\
        #2
    \end{bmatrix*}$
} % two-one matrix

\newcommand{\ttmat}[4]{
    $\begin{bmatrix*}[r]
        #1 & #2 \\
        #3 & #4
    \end{bmatrix*}$
} % two-two matrix

\newcommand{\thmat}[3]{
    $\begin{bmatrix*}[r] 
        #1 \\ 
        #2 \\
        #3 
    \end{bmatrix*}$
} % three-three matrix (line combined form) (specific element in line omitted)

\newcommand{\mytable}[2]{
    \begin{center}
        \begin{tabular}{#1}
            #2
        \end{tabular}
    \end{center}
}

\newcommand{\myalign}[1]{\begin{align*}#1\end{align*}}
\newcommand{\gap}{\vspace*{0.5cm}}
\newcommand{\linearsec}[1]{
    \section*{Question #1}
    \hrule
    \vspace*{0.5cm}
}

\title{Geometry Lab - Computer Graphics \& Robotics - Sept 20 - Blog}
\author{David Cao}
\date{Sept 20}

\begin{document}

\maketitle

\section*{Forward Kinematic Problem \#1}
\section{Problem statement:}
Given a robot with a list of of segments, joints, and prismatic joint, can we give an explicit description of the areas that the robot can reach:
\vspace{3em} \\
\image{0.75}{img1}\\
\begin{center}
  Example 1
\end{center}

\section{Reading summary:}
This week's reading is the same as last week's
\section{Exercise Problem:}
\subsection{Problem 1 statement:}
\image{0.5}{img2}
\subsection{Problem 2 statement:}
\image{0.5}{img3}
\subsection*{Problem 2 statement:}
\section{Discussion and discovery:}
\subsection{Problem 1:}
To represent $cos(\alpha)$ using  polynomials, we can use trigonometric identities:
\myalign{
    & c = cos(\alpha) = cos(\theta_1 +\theta_2 + \theta_3) \\
    & c = cos(\theta_1)cos(\theta_2+\theta_3)-sin(\theta_1)sin(\theta_2+\theta_3) \\
    & c = cos(\theta_1)(cos(\theta_2)cos(\theta_3)-sin(\theta_2)sin(\theta_3))-sin(\theta_1)(sin(\theta_2)cos(\theta_3)+cos(\theta_2)sin(\theta_3)) \\
    & s = sin(\alpha) = sin(\theta_1 + \theta_2 + \theta_3) \\
    & s = sin(\theta_1)cos(\theta_2+\theta_3)+cos(\theta_1)sin(\theta_2+\theta_3) \\
    & s = sin(\theta_1)(cos(\theta_2)cos(\theta_3)-sin(\theta_2)sin(\theta_3))+cos(\theta_1)(sin(\theta_2)cos(\theta_3)+cos(\theta_2)sin(\theta_3)) \\
}
using polynomial:
\myalign{
    c_i = cos(\theta_i) \\
    s_i = sin(\theta_i) \\
    c_i^2 + s_i^2 = 1 \\
}
we can represent $cos(\alpha)$ as:
\myalign{
    c = c_1(c_2c_3-s_2s_3)-s_1(s_2c_3+c_2s_3) \\
    s = s_1(c_2c_3-s_2s_3)+c_1(s_2c_3+c_2s_3) \\
}
\subsection{Problem 2:}
Here's a illustrative graph of problem 2's robot: \\
\begin{center}
    \image{0.15}{img4}
\end{center}
$A_1$ = \mat{
    cos(\theta_1) & -sin(\theta_1) & 0 \\
    sin(\theta_1) & cos(\theta_1) & 0 \\
    0 & 0 & 1
} 
$A_2$ = \mat{
    cos(0) & -sin(0) & l_2 \\
    sin(0) & cos(0) & 0 \\
    0 & 0 & 1
} 
$A_3$ = \mat{
    cos(\theta_2) & -sin(\theta_2) & l_3 \\
    sin(\theta_2) & cos(\theta_2) & 0 \\
    0 & 0 & 1
} \\
\end{document}