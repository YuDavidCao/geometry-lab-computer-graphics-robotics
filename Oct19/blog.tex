\documentclass[11pt]{article}
\usepackage{graphicx}
\usepackage{multirow}
\usepackage{mathtools}
\usepackage{amsmath,amsthm,amssymb, amsfonts}
\usepackage{relsize}
\usepackage{graphicx}
\graphicspath{ {./} }
\usepackage[a4paper, total={6in, 8in}]{geometry}

%description -> general document format:
\newcommand{\boldtext}[1]{\textbf{#1}}

\newcommand{\image}[2]{\includegraphics[scale=#1]{#2}}

\newcommand{\mysection}[1]{
    \section*{#1}
    \hrule
    \vspace*{0.5cm}
}

\newcommand{\mysubsection}[1]{
    \subsection*{· #1}
}

%description -> general math format:
\newcommand{\myunderscore}[2]{
    \underbrace{#1}_\text{#2}
}

\newcommand{\twolinesplit}[3]{
    \begin{equation*}
        #1
        \begin{cases}
             & #2   \\
             & #3
        \end{cases}
    \end{equation*}
}

\newcommand{\threelinesplit}[4]{
    \begin{equation*}
        #1
        \begin{cases}
             & #2   \\
             & #3   \\
             & #4
        \end{cases}
    \end{equation*}
}

%description -> discrete mathematics:
\newcommand{\R}{\mathbb{R}} % Real numbers
\newcommand{\N}{\mathbb{N}} % Natural numbers
\newcommand{\Z}{\mathbb{Z}} % Integers
\newcommand{\Q}{\mathbb{Q}} % Rational numbers
\newcommand{\set}[1]{\{#1\}} % items wrapped in a set
\newcommand{\powerset}[1]{\mathcal{P}(#1)} % items wrapped in a powerset
\newcommand{\floor}[1]{\lfloor #1 \rfloor} % floor notation
\newcommand{\floorfrac}[2]{\lfloor \cfrac{#1}{#2} \rfloor} % floor notation but with a fraction inside
\newcommand{\ceil}[1]{\lceil #1 \rceil} % ceiling notation
\newcommand{\ceilfrac}[2]{\lceil \cfrac{#1}{#2} \rceil} % ceiling notation but with a fraction inside
\newcommand{\Lsum}[2]{\mathlarger{\sum}_{#1}^{#2}} % a larger sigma notation
\newcommand{\hfrac}[2]{\left(\cfrac{#1}{#2}\right)} % a "high" fraction where parathasis's height is adjusted to fit the fraction
\newcommand{\RR}[1]{\mathcal{R}_#1} % testing -> Relational (temporal)
\renewcommand{\R}{\mathcal{R}} % testing -> Relational (temporal)

%description -> linear algebra:

\renewcommand{\det}[1]{
    $\begin{vmatrix*}[r] #1 \end{vmatrix*}$
}

\newcommand{\mateq}{$\backsim$}
\newcommand{\mat}[1]{
    $\begin{bmatrix*}[r]
        #1
    \end{bmatrix*}$
} % general matrix

\newcommand{\tomat}[2]{
    $\begin{bmatrix*}[r]
        #1 \\
        #2
    \end{bmatrix*}$
} % two-one matrix

\newcommand{\ttmat}[4]{
    $\begin{bmatrix*}[r]
        #1 & #2 \\
        #3 & #4
    \end{bmatrix*}$
} % two-two matrix

\newcommand{\thmat}[3]{
    $\begin{bmatrix*}[r] 
        #1 \\ 
        #2 \\
        #3 
    \end{bmatrix*}$
} % three-three matrix (line combined form) (specific element in line omitted)

\newcommand{\mytable}[2]{
    \begin{center}
        \begin{tabular}{#1}
            #2
        \end{tabular}
    \end{center}
}

\newcommand{\myalign}[1]{\begin{align*}#1\end{align*}}
\newcommand{\gap}{\vspace*{0.5cm}}
\newcommand{\linearsec}[1]{
    \section*{Question #1}
    \hrule
    \vspace*{0.5cm}
}

\title{Geometry Lab - Computer Graphics \& Robotics - Oct 19 - Blog}
\author{David Cao}
\date{Oct 19}

\begin{document}

\maketitle

\section*{Inverse Kinematic Problem and Motion Planning \#3}
\section{Problem statement:}
Given a point $(x_1, y_1) = (a, b) \in \mathbb{R}^2$ and an orientation, we wish to determine whether it is possible to place the hand of the robot at that point with that orientation. If it is possible, we wish to find all combinations of joint settings that will accomplish this.

\section{Reading Summary}
In previous readings we have discussed that we can use polynomials to represent the position of the robot arm. \\
\\
Similarly, we can use the polynomial approach to inverse kinematics problem as well. Although often times the 
polynomial equation is difficult to solve due to non-linearity, we can utilize grobner basis to deduce the solution. \\
\\
Consider this problem: \\
\image{0.7}{img1}
Consider the position of the robot hand to be (a, b) and we want to figure out a possible configuration of the arm to reach the position, we can represent the position of the robot arm as a polynomial equation shown below: \\
\image{0.75}{img2}
Given this polynomial we can use grobner basis to simplify the equation and deduce the solution: \\
\image{0.6}{img3}

\section{Exercise Problem:}
\image{0.55}{img4} \\
\image{0.55}{img5} \\
\image{0.55}{img6}

\section{Discussion and discovery:}
\subsection{Problem 2 (b):}
When specifying coordinates, the coordinate is based on the position of the joint. According to the polynomial coordinates:
\myalign{
    & x = cos(\theta) \\
    & y = sin(\theta) \\
    & c_i = cos(\theta_i) \\
    & s_i = sin(\theta_i)
} \\
We can then notice that because when describing the coordinates based on joint2, it is independent from joint 1 because the coordinate is 
based on the position of joint 2 which do not include any components of joint 1. \\
Different from joint 2, when calculating the coordinates based on joint 1, it must also include the coordinate of joint 2 because geometrically
the position of joint 1 is based on the position of joint 2. \\

\subsection{Problem 3:}
The grobner basis when $a = b = 0$ is: \\
\begin{center}
    \image{0.55}{img7}
\end{center}
We notice that the compare to the previous grobner basis, $s_1$ is a free variable which showed that 
there are infinite amount of solutions to this problem. Geometrically, if the robot want to point to the origin
the second robot arm has to point backwards to the exact position of the first arm, shown below: \\
\begin{center}
    \image{0.4}{img8}
\end{center}
\end{document}