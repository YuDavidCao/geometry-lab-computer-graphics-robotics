\documentclass[11pt]{article}
\usepackage{graphicx}
\usepackage{multirow}
\usepackage{mathtools}
\usepackage{amsmath,amsthm,amssymb, amsfonts}
\usepackage{relsize}
\usepackage{graphicx}
\graphicspath{ {./} }
\usepackage[a4paper, total={6in, 8in}]{geometry}

%description -> general document format:
\newcommand{\boldtext}[1]{\textbf{#1}}

\newcommand{\image}[2]{\includegraphics[scale=#1]{#2}}

\newcommand{\mysection}[1]{
    \section*{#1}
    \hrule
    \vspace*{0.5cm}
}

\newcommand{\mysubsection}[1]{
    \subsection*{· #1}
}

%description -> general math format:
\newcommand{\myunderscore}[2]{
    \underbrace{#1}_\text{#2}
}

\newcommand{\twolinesplit}[3]{
    \begin{equation*}
        #1
        \begin{cases}
             & #2   \\
             & #3
        \end{cases}
    \end{equation*}
}

\newcommand{\threelinesplit}[4]{
    \begin{equation*}
        #1
        \begin{cases}
             & #2   \\
             & #3   \\
             & #4
        \end{cases}
    \end{equation*}
}

%description -> discrete mathematics:
\newcommand{\R}{\mathbb{R}} % Real numbers
\newcommand{\N}{\mathbb{N}} % Natural numbers
\newcommand{\Z}{\mathbb{Z}} % Integers
\newcommand{\Q}{\mathbb{Q}} % Rational numbers
\newcommand{\set}[1]{\{#1\}} % items wrapped in a set
\newcommand{\powerset}[1]{\mathcal{P}(#1)} % items wrapped in a powerset
\newcommand{\floor}[1]{\lfloor #1 \rfloor} % floor notation
\newcommand{\floorfrac}[2]{\lfloor \cfrac{#1}{#2} \rfloor} % floor notation but with a fraction inside
\newcommand{\ceil}[1]{\lceil #1 \rceil} % ceiling notation
\newcommand{\ceilfrac}[2]{\lceil \cfrac{#1}{#2} \rceil} % ceiling notation but with a fraction inside
\newcommand{\Lsum}[2]{\mathlarger{\sum}_{#1}^{#2}} % a larger sigma notation
\newcommand{\hfrac}[2]{\left(\cfrac{#1}{#2}\right)} % a "high" fraction where parathasis's height is adjusted to fit the fraction
\newcommand{\RR}[1]{\mathcal{R}_#1} % testing -> Relational (temporal)
\renewcommand{\R}{\mathcal{R}} % testing -> Relational (temporal)

%description -> linear algebra:

\renewcommand{\det}[1]{
    $\begin{vmatrix*}[r] #1 \end{vmatrix*}$
}

\newcommand{\mateq}{$\backsim$}
\newcommand{\mat}[1]{
    $\begin{bmatrix*}[r]
        #1
    \end{bmatrix*}$
} % general matrix

\newcommand{\tomat}[2]{
    $\begin{bmatrix*}[r]
        #1 \\
        #2
    \end{bmatrix*}$
} % two-one matrix

\newcommand{\ttmat}[4]{
    $\begin{bmatrix*}[r]
        #1 & #2 \\
        #3 & #4
    \end{bmatrix*}$
} % two-two matrix

\newcommand{\thmat}[3]{
    $\begin{bmatrix*}[r] 
        #1 \\ 
        #2 \\
        #3 
    \end{bmatrix*}$
} % three-three matrix (line combined form) (specific element in line omitted)

\newcommand{\mytable}[2]{
    \begin{center}
        \begin{tabular}{#1}
            #2
        \end{tabular}
    \end{center}
}

\newcommand{\myalign}[1]{\begin{align*}#1\end{align*}}
\newcommand{\gap}{\vspace*{0.5cm}}
\newcommand{\linearsec}[1]{
    \section*{Question #1}
    \hrule
    \vspace*{0.5cm}
}

\title{Geometry Lab - Computer Graphics \& Robotics - Sept 20 - Blog}
\author{David Cao}
\date{Sept 20}

\begin{document}

\maketitle

\section*{Inverse Kinematic Problem and Motion Planning \#1}
\section{Problem statement:}
Given a point $(x_1, y_1) = (a, b) \in \mathbb{R}^2$ and an orientation, we wish to determine whether it is possible to place the hand of the robot at that point with that orientation. If it is possible, we wish to find all combinations of joint settings that will accomplish this.

\section{Reading summary \& extended details:}
\subsection{Gröbner Basis}
\begin{itemize}
    \item A Gröbner basis is a set of polynomials that can be used to solve systems of polynomial equations.
\end{itemize}
Process of finding a Gröbner basis:
\subsection*{Lexicographical ordering}
\begin{itemize}
    \item The lexicographical ordering of a polynomial is the ordering of the terms in the polynomial by the order of the variables.
    \item For example, the lexicographical ordering of $x + x^2y + 4y^3 + x^3y^2$ according to the order $x < y$ is $x^3y^2 + x^2y + x + 4y^3$
    \item The same equation using a different ordering of $y > x$ will be $4y^3+x^3y^2 + x^2y + x$
\end{itemize}
\subsection*{Buchberger's algorithm}
\begin{itemize}
    \item Buchberger's algorithm is an algorithm that can be used to find a Gröbner basis.
    \item The algorithm is as follows:
    \begin{enumerate}
        \item Given a set of polynomials $f_1, f_2, \dots, f_n$, find the S-polynomial of $f_1$ and $f_2$.
        \item Divide the S-polynomial by the Gröbner basis of the set of polynomials.
        \item If the remainder is not 0, add it to the set of polynomials and repeat the process.
        \item If the remainder is 0, continue to the next pair of polynomials.
    \end{enumerate}
\end{itemize}
This process can be written as a computer algorithm with such pseudocode: \\
\image{0.7}{img2} \\
Clarification:
In the formula for finding the S polynomial, lcm means least common multiple and LT means the leading term \\
\subsection*{Example for finding a S polynomial}
Given current basis $\{y^2-2y+6, -8y\}$, compute the S polynomial of $y^2-2y+6$ and $-8y$: \\
\myalign{
    & S = \cfrac{lcm(LT(p), LT(q))}{LT(p)}p - \cfrac{lcm(LT(p), LT(q))}{LT(q)}q \\
    & LT(p) = y^2, LT(q) = -8y \\
    & lcm(LT(p), LT(q)) = y^2 \\
    & S = \cfrac{y^2}{y^2}(y^2-2y+6) - \cfrac{y^2}{-8y}(-8y) \\
    & S = y^2-2y+6 - y^2 \\
    & S = -2y+6
}
\subsection{Robot Example}
Looking at the example robt below: \\
\image{0.5}{img3}
To simplify this problem we will assume that $l_1 = l_2 = 1$ \\
Based on the polynomial representation of the robot's position discussed in Sept 20, we can write a system of polynomial equations shown below: \\
\begin{center}
    \image{0.5}{img4}
\end{center}
Which by using Buchberger's algorithm, we can find the Gröbner basis of the system of equations shown below: \\
\begin{center}
    \image{0.5}{img5}
\end{center}
Looking at the last equation, we can obtain the following equation: \\
\myalign{
    s = \pm \cfrac{1}{2}\sqrt{(a^2+b^2)(4-(a^2+b^2))}
}
Which we can notice that the equation is only valid if $a^2+b^2 \leq 4$ \\
Geometrically, this make sense because the robot's hand cannot reach a point outside of the circle with radius 2, given that $l_1 = l_2 = 1$ \\
We can also obtain that:
\begin{itemize}
    \item infinitely many distinct settings of joint 1 when $a_2 + b_2 = 0$,
    \item two distinct settings of joint 1 when $0<a_2 +b_2 <4$,
    \item one setting of joint 1 when $a_2 +b_2 =4$,
    \item no possible settings of joint 1 when $a_2 + b_2 > 4$
\end{itemize}

\section{Exercise Problem:}
\image{0.44}{img6}

\section{Discussion and discovery:}

\end{document}